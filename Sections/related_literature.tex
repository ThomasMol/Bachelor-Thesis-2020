\subsection{Job Interviews \& Job Application Videos}
Many organizations and corporations utilize job interviews to select the most qualified person for an open position. These interviews typically consist of a short meeting where the applicant is assessed based on their skills and experience but also on their personality, mood, energy and motivation during the interview. These dimensions are assessed to determine and predict if the applicant would be successful at the job they have applied for. Hiring an incompetent worker would not only mean the loss of salary but also the time it takes to search for a new candidate. Despite this risk, job interviews prove to be a reliable method in the job candidate selection process (\cite{weekley1987reliability}). 

However, with the increase in use of online communication tools and online application processes job the approach for job interviews has changed as well. Many organizations choose to use online platforms to recruit new employees for open positions. A study in 2011 showed that 76\% of unemployed people search online for new jobs (\cite{faberman2016does}), showing how reliant organizations have become on online job recruitment. Research has been done to improve the online recruitment process by increasing market transparency and lower transaction cost with the use of semantic web technologies (\cite{bizer2005impact}).  

\subsection{Apparent Personality Traits}
In psychology, the characterization of someone's personality is based on long-term behavior. Personalities have sophisticated structures and are constructed of many different aspects such as habits and the way people think or feel. Therefore it can be difficult to analyze, measure and categories personalities to divide humans into different personality types. Instead, psychology now mainly focuses on personality traits, where the OCEAN (or CANOE) Big Five personality trait model developed by Paul Costa and Robert R. McCrae is most widely accepted. These traits are; Openness to experience, Conscientiousness, Extraversion, Agreeableness and lastly, Neuroticism. 

The first factor of the 5 factor personality trait model is Openness to experience. People who score high in this factor generally are imaginative and have a complex emotional life. They are also artistically sensitive while low scores are considered simple and narrow-minded. The Conscientiousness factor is related to how well-organized and persistent a person is. People with low conscientiousness are considered careless, undependable and disorganized. The traits underlying the Extraversion dimension are related to sociability and activity. Extraverted people are more talkative, adventurous and sociable while introverted people tend to be more silent, cautious and reclusive. Agreeableness is a dimension of interpersonal behavior. People with high scores on agreeableness are cooperative, good natured and are seen as friendly. Low scores are more irritable, jealous and stubborn. Lastly, the Neuroticism dimension represents a person's tendency to experience emotional anxiety. The dimension can also be described as someone's emotional stability. People with high neuroticism are more anxious and nervous and tend to worry more while people with low scores are more calm and composed. Although having a low neuroticism score seems more desirable, low scores can also be too optimistic(?) and act carelessly. 

Several works have used this model to analyze human behavior like \textcite{quercia2011our} who analyzed relationships with different types of Twitter users. Or \textcite{allbeck2008creating}, who used the OCEAN model to map personalities to crowd simulations in order to improve the realism of their simulations. 

Another approach to apply the five factor model is to annotate the apparent personality traits (\cite{junior2018first}, \cite{chen2016overcoming}). The annotator has to annotate the impression subject leaves rather than the actual personality. This is easier as the annotator relies on external evaluations without any direct involvement of the subject. 

\subsection{Mood Primitives \& Likeability}
Another dimension that can be used to classify a person's behavior is to assess their mood or emotions over a short period of time. While the terms emotion and mood seem to be used interchangeably, they are in fact different. In psychology, an emotion is seen as a immediate reaction to a stimulus, while a mood lasts for a longer period (\cite{bower2000affectmemory}). For example, someone can be in positive mood throughout the day but still have a negative emotional reaction to a stimulus like a bad smell, or being insulted (\cite{matlin2012cognition}). Additionally, this makes is more difficult to identify and specify the cause of someone's mood as there is no direct cause like in a emotional reaction (\cite{desmet2016mood}).

In order to simplify the classification of a subject's mood, emotion classification can be used. Instead of trying to determine someone's mood over a short period, the emotions can be used as an indicator or their mood. To do this, a base set of emotions is needed to start classifying a subject's mood. The six basic emotions developed by Paul Ekman in 1992 are widely regarded as the standard to identify emotions (\cite{ekman1992argument}). Ekman found that there are nine distinct characteristics that distinguish the basic emotions, for example, the physiological response, the duration and their quick onset. He also found that cultural background had no impact in the identification of emotion. Meaning that the six basic emotion can be used across different cultures to classify emotions. The six emotion that Ekman identified as basic are; anger, happiness, surprise, disgust, sadness and fear (\cite{ekman1992argument}). 

Research has also indicated that mood can be defined as a dimensional model with two dimensions: valence and arousal. This model was first proposed by Russel in 1980 (\cite{russell1980circumplex}), and is known as the circumplex model of affect. In this model arousal can be seen as the amount of energy a person shows. This dimension ranges from a deactivated state, described as sleepiness, to an activated state, which can be described as aroused. The valence dimension refers to the pleasantness or unpleasantness. This dimension typically ranges from a state of misery to a state of pleasure. Combining these two dimension creates four quadrants with four basic moods; angry, happy, sad and relaxed. 