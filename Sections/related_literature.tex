In psychology, the characterisation of someone's personality is based on long-term behaviour. Personalities have sophisticated structures and are constructed of many different aspects such as habits and the way people think or feel. Therefore it can be difficult to analyse, measure and categorise personalities to divide humans into different personality types. Instead, psychology now mainly focuses on personality traits, where the OCEAN (or CANOE) Big Five personality trait model developed by Paul Costa and Robert R. McCrae is most widely accepted. These traits are; Openness to experience, Conscientiousness, Extraversion, Agreeableness and lastly, Neuroticism. 

The first factor of the 5 factor personality trait model is Openness to experience. People who score high in this factor generally are imaginative and have a complex emotional life. They are also artistically sensitive while low scores are considered simple and narrow-minded. The Conscientiousness factor is related to how well-organised and persistent a person is. People with low conscientiousness are considered careless, undependable and disorganised. The traits underlying the Extraversion dimension are related to sociability and activity. Extraverted people are more talkative, adventurous and sociable while introverted people tend to be more silent, cautious and reclusive. Agreeableness is a dimension of interpersonal behaviour. People with high scores on agreeableness are cooperative, good natured and are seen as friendly. Low scores are more irritable, jealous and stubborn. Lastly, the Neuroticism dimension represents a person's tendency to experience emotional anxiety. The dimension can also be described as someone's emotional stability. People with high neuroticism are more anxious and nervous and tend to worry more while people with low scores are more calm and composed. Although having a low neuroticism score seems more desirable, low scores can also be too optimistic and act carelessly (bron). 

Several works have used this model to analyse human behaviour like \cite{quercia2011our} who analysed relationships with different types of Twitter users. Or \cite{allbeck2008creating} who used the OCEAN model to map personalities to crowd simulations in order to improve the realism of their simulations. 
