Affective computing and artificial intelligence are progressively becoming more popular and widely adopted by organizations (\cite{davenport2018artificial}, \cite{tao2005affective}). Affective computing systems intent to be responsive, interpret and recognize human affects, e.g., emotions. Many applications of such affective systems can be developed for educational tools, mental healthcare and much more. Furthermore, artificial intelligence and machine learning can be utilized in the development of affective systems. The performance capability of today's computer systems are becoming increasingly better to the point at which affective systems can be used in real-time. 

Another area where affective computing and machine learning methods can be applied is in the job candidate selection process. Job interviews are an essential part of the job candidate selection process for many corporations and organizations. Affective computing could help both the job seeker and the job recruiter. In this thesis we will explore a system that can predict whether an applicant should be invited to a job interview based on their perceived personality and mood. This system will be modeled based on video modality. Additionally, the system can help the job recruiter visualize biases in the selection process. The job seeker could also benefit from such a system as the system could show them what impression they give the recruiter and even suggest improvements to increase their chances to be invited to the job interview. 

Nevertheless, caution should be taken when developing and using such a system. It may seem like a game changer; a system that could automatically predict viable job candidates, especially when there is a staggering amount of candidates. However, supervised learning will be needed to create such a system. This means the system will rely on human based annotations for its supervision. As a result, the system will also learn the biases and stereotypes that annotators might have. For example, a younger age group is given higher overall job interview ratings compared to an older age group (\cite{morgeson2008review}). However, in actual job performance, the older workers are perceived to have better performance (\cite{truxillo2012perceptions}). Therefore, first impressions may not be a robust basis to determine actual job performance of a candidate.

In this thesis we will first discuss related literature on apparent personality traits, mood primitives and video resumes and job interview. Then, we will discuss our proposed methodology for our experiments which includes data annotation, feature extraction from video clips, feature normalization and lastly, classification and explainability models. We will also analyze the inter-rater agreement as we will have two annotators annotating our data-set. Next, we will report and discuss the results of the proposed experiments. After, we will look at the explainability analysis of the models. Lastly we will discuss our findings, shortcomings and describe future research than can be done regarding this research area. 
