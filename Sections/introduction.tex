Have you ever wondered why computers seem so emotionless? Well that's because most computers cannot sense your emotions and react to them with convincing emotion. If we want computers to behave politely, human-like and social, computers need to have some sort of common-sense reasoning. They need to be able to understand a user's emotional state, current mood and maybe even personality type. Another feature they should have is the ability to respond with emotion. A computer will seem more genuine and convincing if they can appropriately express emotions. These research topics are what affective computing is all about. Today's computers are powerful enough, as in computational power, that real-time tracking of a user's emotions is increasingly becoming more feasible. For example, a camera app on a smartphone can, in real-time, detect faces and people smiling. 

Another area where affective computing can be applied is in a job candidate selection process. Job interviews are an essential part of the job candidate selection process for many corporations and organizations. Affective computing could help both the job seeker and the job recruiter. A well developed system can analyze videos that are submitted by the job candidates to predict whether the candidate should be invited for an interview. Additionally, the system can help the job recruiter visualize biases in the selection process. The job seeker could also benefit from such a system as the system could show them what impression they give the recruiter and even suggest improvements to increase their chances to be invited to the job interview. 

Nevertheless, caution should be taken when developing and using such a system. It may seem like a game changer; a system that could automatically predict viable job candidates, especially when there is a staggering amount of candidates. However, supervised learning will be needed to create such a system. This means the system will rely on human based annotations for its supervision. As a result the system will also learn the biases and stereotypes that annotators have. For example, a younger age group is given higher overall interview ratings compared to an older age group. However, in actual job performance, the older workers are perceived to have better performance. Therefore, first impressions are not a robust basis to determine actual job performance of a candidate.